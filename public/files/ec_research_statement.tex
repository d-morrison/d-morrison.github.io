\documentclass[12pt]{article}
\title{Research Statement} 
\author{Emilie Campos} 
\date{\today}

\usepackage[margin=1in]{geometry}

\begin{document}
\maketitle
As a researcher in the highly collaborative field of biostatistics, I believe that methods development should be complemented by strong communication practices. There is no stronger form of communication for a biostatistician than clear and concise data visualization that provides clincial researchers and other interested parties tools for assessing the usefulness of novel statistical methods. My passion for both data visualization and statisical methods development inspires my creation of R Shiny apps and packages that parallel my quantitative contributions. 

I am interested in studying the heterogeneous nature of Autism Spectrum Disorder and Attention Deficit Hyperactivity Disorder in children. Using electroencephalograms, we are able to begin to compare the neurological development of children with these disorders through event-related potential analysis. ERP waveforms are the summation of many unknown overlapping signals. Due to this summation, changes in the peak or mean amplitude of a waveform over a given time period cannot be attributed to a particular ERP component of ex ante interest, as is the standard approach to ERP analysis. Though this problem is widely recognized, solutions have remained out of reach and the problem is largely neglected in practice. Through data reduction techniques that reveal the underlying basis fucntions for ERP waveforms, we aim to develop an algorithm that changes the standard approach for analyzing ERP data and reveal the neurological mechanisms affected in those with psychiatric disorders.

I became a biostatistician to use my strong mathematical background to build bridges between biological research and proper statistical practice. It is my hope to continue not only developing statistical methods for the analysis of multi-task EEG data but also to contribute to the research community as a whole by creating interactive applications and R packages based on my research.
\end{document}